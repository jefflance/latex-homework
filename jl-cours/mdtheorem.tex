%%%%%%%%%%%%%%%%%%%%%%%%%%%%%%%%%%%%%%%%%%%%%%%%%%%%%%%%%%%%%%%%%%%%%%%%%%%%%%%%
%
%
% MDTHEOREM
%
%


%% New theorem declarations and redirection of all standard previously defined
%% environments to their starred version as they don't display numbering.
%%

% definition
\mdtheorem[style=pencil, theoremseparator={},
           theoremspace={}, roundcorner=6pt]{@def1}{\Jd Définition}
\mdtheorem[style=nologo, theoremseparator={},
           theoremspace={}, roundcorner=6pt]{@def2}{\Jd Définition}

\RenewDocumentEnvironment{@def1}{ o } {%begin
  \IfNoValueTF {#1} {
    \begin{@def1*}
  }{
    \begin{@def1*}[\hfill{\normalfont\slshape#1}]
  }
}{%end
  \end{@def1*}
}

\RenewDocumentEnvironment{@def2}{ o } {%begin
  \IfNoValueTF {#1} {
    \begin{@def2*}
  }{
    \begin{@def2*}[\hfill{\normalfont\slshape#1}]
  }
}{%end
  \end{@def2*}
}

% theoreme
\mdtheorem[style=pencil, theoremseparator={},
           theoremspace={}, roundcorner=6pt,
           outerlinewidth=1pt, outerlinecolor=black]{@thm1}{\Jd Théorème}
\mdtheorem[style=nologo, theoremseparator={},
           theoremspace={}, roundcorner=6pt,
           outerlinewidth=1pt, outerlinecolor=black]{@thm2}{\Jd Théorème}

\RenewDocumentEnvironment{@thm1}{ o } {%begin
  \IfNoValueTF {#1} {
    \begin{@thm1*}
  }{
    \begin{@thm1*}[\hfill{\normalfont\slshape#1}]
  }
}{%end
  \end{@thm1*}
}

\RenewDocumentEnvironment{@thm2}{ o } {%begin
  \IfNoValueTF {#1} {
    \begin{@thm2*}
  }{
    \begin{@thm2*}[\hfill{\normalfont\slshape#1}]
  }
}{%end
  \end{@thm2*}
}

\RenewDocumentEnvironment{@coro1}{ o } {%begin
  \IfNoValueTF {#1} {
    \begin{@coro1*}
  }{
    \begin{@coro1*}[\hfill{\normalfont\slshape#1}]
  }
}{%end
  \end{@coro1*}
}

\RenewDocumentEnvironment{@coro2}{ o } {%begin
  \IfNoValueTF {#1} {
    \begin{@coro2*}
  }{
    \begin{@coro2*}[\hfill{\normalfont\slshape#1}]
  }
}{%end
  \end{@coro2*}
}

% propriete
\mdtheorem[style=pencil, theoremseparator={}, theoremspace={},
           outerlinewidth=1pt, outerlinecolor=black]{@ppt1}{\Jd Propriété}
\mdtheorem[style=nologo, theoremseparator={}, theoremspace={},
           outerlinewidth=1pt, outerlinecolor=black]{@ppt2}{\Jd Propriété}

\RenewDocumentEnvironment{@ppt1}{ o } {%begin
  \IfNoValueTF {#1} {
    \begin{@ppt1*}
  }{
    \begin{@ppt1*}[\hfill{\normalfont\slshape#1}]
  }
}{%end
  \end{@ppt1*}
}

\RenewDocumentEnvironment{@ppt2}{ o } {%begin
  \IfNoValueTF {#1} {
    \begin{@ppt2*}
  }{
    \begin{@ppt2*}[\hfill{\normalfont\slshape#1}]
  }
}{%end
  \end{@ppt2*}
}

% demonstration
\mdtheorem[style=pencil, theoremseparator={},
           theoremspace={}]{@dem1}{\Jd Démonstration}
\mdtheorem[style=nologo, theoremseparator={},
           theoremspace={}]{@dem2}{\Jd Démonstration}

\RenewDocumentEnvironment{@dem1}{ o } {%begin
  \IfNoValueTF {#1} {
    \begin{@dem1*}
  }{
    \begin{@dem1*}[\hfill{\normalfont\slshape#1}]
  }
}{%end
  \end{@dem1*}
}

\RenewDocumentEnvironment{@dem2}{ o } {%begin
  \IfNoValueTF {#1} {
    \begin{@dem2*}
  }{
    \begin{@dem2*}[\hfill{\normalfont\slshape#1}]
  }
}{%end
  \end{@dem2*}
}

% consequence
\mdtheorem[style=nologo, theoremseparator={}, theoremspace={},
           outerlinewidth=1pt, outerlinecolor=black]{@csq1}{\Jd Conséquence(s)}
\mdtheorem[style=nologo, theoremseparator={}, theoremspace={},
           outerlinewidth=1pt, outerlinecolor=black]{@csq2}{}

\RenewDocumentEnvironment{@csq1}{ o } {%begin
  \IfNoValueTF {#1} {
    \begin{@csq1*}
  }{
    \begin{@csq1*}[\hfill{\normalfont\slshape#1}]
  }
}{%end
  \end{@csq1*}
}

\RenewDocumentEnvironment{@csq2}{ o } {%begin
  \IfNoValueTF {#1} {
    \begin{@csq2*}
  }{
    \begin{@csq2*}[\hfill{\normalfont\slshape#1}]
  }
}{%end
  \end{@csq2*}
}

% vocabulaire
\mdtheorem[style=noborder, theoremseparator={},
           theoremspace={}]{@voc1}{\Jd Vocabulaire}
\mdtheorem[style=nologo, theoremseparator={},
          theoremspace={}]{@voc2}{\Jd Vocabulaire}

\RenewDocumentEnvironment{@voc1}{ o } {%begin
  \IfNoValueTF {#1} {
    \begin{@voc1*}
  }{
    \begin{@voc1*}[\hfill{\normalfont\slshape#1}]
  }
}{%end
  \end{@voc1*}
}

\RenewDocumentEnvironment{@voc2}{ o } {%begin
  \IfNoValueTF {#1} {
    \begin{@voc2*}
  }{
    \begin{@voc2*}[\hfill{\normalfont\slshape#1}]
  }
}{%end
  \end{@voc2*}
}

% commentaire
\mdtheorem[style=comment, frametitle=none,
           frametitleaboveskip=0em, frametitlebelowskip=0em]{@com}{}

\RenewDocumentEnvironment{@com}{ o } {%begin
 \IfNoValueTF {#1} {
   \begin{@com*}
 }{
   \begin{@com*}[#1]
 }
}{%end
  \end{@com*}
}

% remarque
\mdtheorem[style=warning, theoremseparator={ -},
           theoremspace={}]{@rmq1}{\Jd Remarque(s)}
\mdtheorem[style=warning, theoremseparator={},
           theoremspace={}]{@rmq2}{}

\RenewDocumentEnvironment{@rmq1}{ o } {%begin
  \IfNoValueTF {#1} {
    \begin{@rmq1*}
  }{
    \begin{@rmq1*}[#1]
  }
}{%end
  \end{@rmq1*}
}

\RenewDocumentEnvironment{@rmq2}{ o } {%begin
  \IfNoValueTF {#1} {
    \begin{@rmq2*}
  }{
    \begin{@rmq2*}[#1]
  }
}{%end
  \end{@rmq2*}
}

% exemple
\mdtheorem[style=eye, theoremseparator={ -}, theoremspace={}]{@exe1}{\Jd Exemple(s)}
\mdtheorem[style=eye, theoremseparator={}, theoremspace={}]{@exe2}{}

\RenewDocumentEnvironment{@exe1}{ o } {%begin
  \IfNoValueTF {#1} {
    \begin{@exe1*}
  }{
    \begin{@exe1*}[#1]
  }
}{%end
  \end{@exe1*}
}

\RenewDocumentEnvironment{@exe2}{ o } {%begin
  \IfNoValueTF {#1} {
    \begin{@exe2*}
  }{
    \begin{@exe2*}[#1]
  }
}{%end
  \end{@exe2*}
}

% application
\mdtheorem[style=cogs, theoremseparator={},
           theoremspace={}]{@app1}{\Jd Application}
\mdtheorem[style=cogs, theoremseparator={},
          theoremspace={}]{@app2}{}

\RenewDocumentEnvironment{@app1}{ o } {%begin
  \IfNoValueTF {#1} {
    \begin{@app1*}
  }{
    \begin{@app1*}[\hfill{\normalfont\slshape#1}]
  }
}{%end
\end{@app1*}
}

\RenewDocumentEnvironment{@app2}{ o } {%begin
  \IfNoValueTF {#1} {
    \begin{@app2*}
  }{
    \begin{@app2*}[\hfill{\normalfont\slshape#1}]
  }
}{%end
  \end{@app2*}
}

% methode
\mdtheorem[style=bookmark, theoremseparator={ -},
           theoremspace={}]{@met1}{\Jd Méthode}
\mdtheorem[style=noborder, theoremseparator={},
           theoremspace={}]{@met2}{}

\RenewDocumentEnvironment{@met1}{ o } {%begin
  \IfNoValueTF {#1} {
    \begin{@met1*}
  }{
    \begin{@met1*}[#1]
  }
}{%end
  \end{@met1*}
}

\RenewDocumentEnvironment{@met2}{ o } {%begin
  \IfNoValueTF {#1} {
    \begin{@met2*}
  }{
    \begin{@met2*}[#1]
  }
}{%end
  \end{@met2*}
}

% Need to adapt the code in order to get numbering and
% framing
%% application
%\newtheoremstyle{@app1}
%  {\topsep}% espace avant
%  {\topsep}% espace apres
%  {}% Police utilisee par le style de thm
%  {}% Indentation (vide = aucune, \parindent = indentation paragraphe)
%  {\bfseries}% Police du titre de thm
%  {}% Signe de ponctuation apres le titre du thm
%  {\newline}% Espace apres le titre du thm (\newline = linebreak)
%  {\thmname{#1}\thmnumber{ #2}\thmnote{ - \normalfont{\textit{#3}}}}
%  % composants du titre du thm : \thmname = nom, \thmnumber = numéro, \thmnote = sous-titre
%
%\theoremstyle{@app1}
%\newtheorem{application1}{\Jd Application}
%
%% application
%\newtheoremstyle{@app2}
%{\topsep}% espace avant
%{\topsep}% espace apres
%{}% Police utilisee par le style de thm
%{}% Indentation (vide = aucune, \parindent = indentation paragraphe)
%{\bfseries}% Police du titre de thm
%{}% Signe de ponctuation apres le titre du thm
%{\newline}% Espace apres le titre du thm (\newline = linebreak)
%{\thmname{#1}\thmnote{ - \normalfont{\textit{#2}}}}
%% composants du titre du thm : \thmname = nom, \thmnumber = numéro, \thmnote = sous-titre
%
%\theoremstyle{@app2}
%\newtheorem*{application2}{\Jd Application}
%
%\counterwithin*{application1}{section}

% exercice
\newtheoremstyle{@exo1}
  {\topsep}% espace avant
  {\topsep}% espace apres
  {}% Police utilisee par le style de thm
  {}% Indentation (vide = aucune, \parindent = indentation paragraphe)
  {\bfseries}% Police du titre de thm
  {}% Signe de ponctuation apres le titre du thm
  {\newline}% Espace apres le titre du thm (\newline = linebreak)
  {\thmname{#1}\thmnumber{ #2}\thmnote{ - \normalfont{\textit{#3}}}}
  % composants du titre du thm : \thmname = nom, \thmnumber = numéro, \thmnote = sous-titre

\theoremstyle{@exo1}
\newtheorem{exercice1}{\Jd Exercice}

% exercice
\newtheoremstyle{@exo2}
{\topsep}% espace avant
{\topsep}% espace apres
{}% Police utilisee par le style de thm
{}% Indentation (vide = aucune, \parindent = indentation paragraphe)
{\bfseries}% Police du titre de thm
{}% Signe de ponctuation apres le titre du thm
{\newline}% Espace apres le titre du thm (\newline = linebreak)
{\thmname{#1}\thmnote{ - \normalfont{\textit{#2}}}}
% composants du titre du thm : \thmname = nom, \thmnumber = numéro, \thmnote = sous-titre

\theoremstyle{@exo2}
\newtheorem*{exercice2}{\Jd Exercice}

\counterwithin*{exercice1}{section}
%%%%%%%%%%%%%%%%%%%%%%%%%%%%%%%%%%%%%%%%%%%%%%%%%%%%%%%%%%%%%%%%%%%%%%%%%%%%%%%%
